% \documentclass{article}
% \usepackage{pgfplots}
% \pgfplotsset{compat=1.18}
% \usepgfplotslibrary{groupplots} % Load groupplots library

% \begin{document}

% \begin{figure}[htbp]
    % \centering
    \begin{tikzpicture}
    
    % Group style: two plots side by side, wider and with one common legend
    \begin{groupplot}[
        group style={
            group size=2 by 1,
            horizontal sep=.5cm, % Horizontal space between the plots
        },
        width=10cm, % Increase the width of each plot
        height=8cm,
        xlabel={$m$},
        ylabel={Time (microseconds)}, % Left plot has a y-axis label
        grid=major,
        % legend to name=commonlegend, % Single common legend
        legend style={at={(0.25,0.98)}, anchor=north}
        ]
    
    % First subplot (log scale, x <= 12)
    \nextgroupplot[
        ymode=log,
        xmax=18,
        xmin=4
    ]
    
\addplot table [x=Range, y={Mean CPU Time (microseconds)}, col sep=comma] {Benchmark/BM_libiop_additive_fft.csv};
    \addplot table [x=Range, y={Mean CPU Time (microseconds)}, col sep=comma] {Benchmark/BM_gao_additive_fft_lvl1.csv};
    \addplot table [x=Range, y={Mean CPU Time (microseconds)}, col sep=comma] {Benchmark/BM_gao_additive_fft_lvl2.csv};
    \addplot table [x=Range, y={Mean CPU Time (microseconds)}, col sep=comma] {Benchmark/BM_gao_additive_fft_co.csv};
    \addplot [red, mark=diamond*] table [x=Range, y={Mean CPU Time (microseconds)}, col sep=comma] {Benchmark/BM_gao_additive_fft_co_lvl2.csv};
    \addplot table [x=Range, y={Mean CPU Time (microseconds)}, col sep=comma] {Benchmark/BM_cantor_additive_fft.csv};
    \addplot table [x=Range, y={Mean CPU Time (microseconds)}, col sep=comma] {Benchmark/BM_cantor_additive_fft_precmp.csv};

    % Second subplot (regular scale, x >= 12, without y-axis label)
    \nextgroupplot[
        xmin=18,
        xmax=28,
        % yticklabel=\empty, % Remove y-axis labels on the right plot
        ylabel={Time (microseconds)}, % Remove the y-axis label
        axis y line=right, % Enable the right-side y-axis
        ylabel near ticks % Optional: move y-axis label near the ticks
    ]

        \addplot table [x=Range, y={Mean CPU Time (microseconds)}, col sep=comma] {Benchmark/BM_libiop_additive_fft.csv};
    \addlegendentry{GM \cite{libiop}};

    \addplot table [x=Range, y={Mean CPU Time (microseconds)}, col sep=comma] {Benchmark/BM_gao_additive_fft_lvl1.csv};
    \addlegendentry{GM PCL1};

    \addplot table [x=Range, y={Mean CPU Time (microseconds)}, col sep=comma] {Benchmark/BM_gao_additive_fft_lvl2.csv};
    \addlegendentry{GM PCL2};

    \addplot table [x=Range, y={Mean CPU Time (microseconds)}, col sep=comma] {Benchmark/BM_gao_additive_fft_co.csv};
    \addlegendentry{GM CO};

    \addplot [red, mark=diamond*] table [x=Range, y={Mean CPU Time (microseconds)}, col sep=comma] {Benchmark/BM_gao_additive_fft_co_lvl2.csv};
    \addlegendentry{GM CO PCL2};

    \addplot table [x=Range, y={Mean CPU Time (microseconds)}, col sep=comma] {Benchmark/BM_cantor_additive_fft.csv};
    \addlegendentry{Cantor};

    \addplot table [x=Range, y={Mean CPU Time (microseconds)}, col sep=comma] {Benchmark/BM_cantor_additive_fft_precmp.csv};
    \addlegendentry{Cantor PC};
    

    \end{groupplot}
    
    % Place the common legend below both plots
    % \node at ($(group c1r1.south east)!0.5!(group c2r1.south west)$) {\pgfplotslegendfromname{commonlegend}};
    
    \end{tikzpicture}
    % \caption{Performance comparison of different additive FFT algorithms}
% \end{figure}

% \end{document}
