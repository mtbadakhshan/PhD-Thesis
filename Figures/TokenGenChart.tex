\documentclass[tikz]{standalone}
\usepackage{tikz}
\usepackage{amsmath}
\usetikzlibrary{shapes.geometric, arrows}
\begin{document}

\tikzstyle{crh} = [rectangle, rounded corners, minimum width=2cm, minimum height=1cm,text centered, draw=black]
\tikzstyle{value} = [rectangle, minimum width=1cm, minimum height=1cm,text centered]
\tikzstyle{arrow} = [thick,->,>=stealth]

\begin{tikzpicture} [node distance=5cm]
  \node (crh1) [crh] {\Large $\mathcal{H}$ };
  \node (crh2) [crh, left of=crh1, xshift=-2cm] {\Large \textsf{COMM}};
  \node (rho) [value, below of=crh1, yshift=3.5cm] {\LARGE $\rho$};
  \node (PKsig) [value, below of=crh2, yshift=2.5cm] {\LARGE PKsig};
  \node (q) [value, left of=PKsig, xshift=3cm, yshift=1cm] {\LARGE $q$};
  \node (eol) [value, above of=crh1, yshift=-3cm] {\LARGE \texttt{eol}};
  \node (cm) [value, above of=crh2, yshift=-3cm] {\LARGE \texttt{cm}};
  

  \draw [arrow] (rho) -- (crh1);
  \draw [arrow] (rho) -| ([xshift=1.5cm] crh2);
  \draw [arrow] (PKsig) -- (crh2);
  \draw [arrow] (q) -| ([xshift=-1.5cm] crh2);
  \draw [arrow] (crh1) -- (eol);
  \draw [arrow] (crh2) -- (cm);
  %%%
\end{tikzpicture}
\end{document}