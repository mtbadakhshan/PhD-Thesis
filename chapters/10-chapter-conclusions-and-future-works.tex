%======================================================================
\chapter{Conclusions and Future Works} \label{ch:conclusion}
%======================================================================


\section{Concluding Remarks}
This thesis conducted a comprehensive study on optimizing and enhancing the performance of post-quantum secure zero-knowledge succinct non-interactive arguments of knowledge (\glspl{zksnark}). As \glspl{zksnark} play an increasingly vital role in privacy-preserving applications, blockchain scalability, and cryptographic proofs of data integrity, improving their efficiency is paramount. A central focus was placed on additive Fast Fourier Transform (\gls{fft}) algorithms, which are foundational to post-quantum secure zkSNARK constructions. The thesis also introduced Zupply, a proof-of-concept framework that demonstrates the practical use of \gls{zksnark} protocols for building privacy-preserving, decentralized applications on public blockchain platforms.


Chapters~\ref{ch:Intro}, \ref{ch:preliminaries}, and \ref{ch:lit_review} provided the introduction and background necessary to understand zero-knowledge proofs (\glspl{zkp}), \glspl{zksnark}, and explored the design of additive \gls{fft} algorithms. Chapter~\ref{ch:lit_review} also reviews existing challenges and current solutions in supply chain management (\gls{scm}) systems. It highlighted the critical need for a privacy-preserving framework to support \gls{scm} systems deployed over permissionless blockchains.

Chapter~\ref{ch:additive-fft} demonstrated the effectiveness of leveraging the Cantor special basis~\cite{Cantor1989FFT} to enable the use of the Cantor additive \gls{fft} algorithm in post-quantum secure \glspl{zksnark}, with a particular focus on the Aurora protocol~\cite{Aurora2019}. The implementation showed that replacing the Gao-Mateer \gls{fft}~\cite{Gao2010FFT} with the Cantor \gls{fft} resulted in a substantial reduction in computation time. In addition, this Chapter presented a detailed theoretical analysis of the Cantor \gls{fft}'s computational cost, including exact counts of additions and multiplications. It also evaluated the \gls{fft} call complexity arising during the encoding of the  rank 1 constraint system (\gls{r1cs}) in Aurora, with respect to the number of constraints, variables, and the chosen security parameter. Additionally, this chapter introduced optimized building blocks for the Cantor \gls{fft} implementation and proposed precomputation techniques that reduced overhead in both the Cantor and Gao-Mateer \gls{fft} algorithms when the basis of the affine subspaces was predetermined.

Chapter~\ref{ch:polaris} presented an instantiation of the fast reed-solomon interactive oracle proofs of proximity (\gls{fri}) protocol~\cite{FRI2018} that eliminated field inversion operations in both the Commit and Query phases, contributing to improved efficiency. It also introduced a tailored instantiation of the GKR circuit designed to minimize the number of gates, with the goal of reducing communication overhead as well as the computational complexities of both the verifier and the prover.

Chapter~\ref{ch:zupply-design} presented the Zupply framework, which offers an anonymous, unlinkable, trustless, fully decentralized, and efficient solution for managing off-chain, directed acyclic graph (\gls{dag}) structured data in supply chains. The framework introduced an anonymous authentication token (\gls{aat}) scheme, including the \gls{aat} ownership transfer (\gls{aatot}) protocol, realized through the \textsf{OT-Protocol}, to ensure unlinkability during ownership transfers. Its efficiency is achieved via the \textsf{MHT-Protocol}, which updates the Merkle hash tree (\textsf{MHT}) root on the contract $\mathcal{C}_Z$, eliminating the need to store the entire tree on-chain. By supporting off-chain data storage with anonymous authentication, the framework addresses the high cost of on-chain storage\footnote{For example, storing 1 MB of data on Ethereum exceeds USD 38,000 at current ETH prices.}. This flexibility enables customizable storage strategies while preserving user anonymity.

Chapter~\ref{ch:zupply_implementation} presented the implementation of the Zupply framework. The framework was implemented in \CC and Solidity, initially using the Groth16 \gls{zksnark}~\cite{Groth2016} over the BN254 curve~\cite{BNcurve}. This implementation demonstrated efficiency despite the high gas costs on Ethereum, although its security level is limited to approximately 100 bits~\cite{Barbulescu2019}. To improve security, the BLS12-381 curve~\cite{BLS_curve2003} was also used, offering 128-bit security. Furthermore, replacing the Groth16 \gls{zksnark} with Aurora~\cite{Aurora2019} enhanced the framework by providing plausible post-quantum security against quantum-enabled malicious provers attempting to forge invalid proofs. It also enabled the system to operate without requiring a trusted setup, either at initialization or during future updates. Such updates may include increasing the number of \textsf{MHT} layers or modifying the structure of \glspl{aat}.







\section{Future Works}

Looking ahead, several promising research directions emerge.  Building on the results and optimizations presented in Chapter~\ref{ch:additive-fft}, several promising directions for future research can be explored. One such direction is to investigate the LCH additive \gls{fft} \cite{LCH-FFT2016}, including its basis conversion in conjunction with the Cantor special basis, and explore optimizations for other components of Aurora, such as the FRI protocol~\cite{FRI2018}. Another avenue could involve applying tower field constructions, or using  \gls{fft} algorithms to accelerate field multiplications, where the \texttt{CLMUL} instruction is not available on CPUs. Moreover, extending the additive \gls{fft} optimizations to support parallelism on hardware accelerators (e.g., GPUs and FPGAs) could further boost performance. Additionally, the proposed optimizations may also be applied to other post-quantum secure \glspl{zksnark} operating over binary extension fields, such as Fractal~\cite{Chiesa2020Fractal}, STARK~\cite{Ben-Sasson2018STARK}, Ligero~\cite{Ames2017Ligero}, etc. In addition, integrating the proposed optimizations into emerging zkEVMs and rollup architectures, such as \cite{STARKnet, PolygonZKEVM, zkSync} will be key to scaling computation in future decentralized applications. Also, this can be used in post-quantum secure digital signature algorithms such as Preon~\cite{Preon2023}.

	
In future work, a full implementation of the Polaris \gls{zksnark}, accompanied by comprehensive performance benchmarks and an evaluation of the expected improvements, will be presented as part of the complete realization of the Polaris protocol. Future work can investigate alternative Reed–Solomon (\gls{rs}) proximity testings that have less query complexity, such as STIR~\cite{Arnon2024STIR}, or faster verification algorithm, such as WHIR~\cite{Arnon2024WHIR}.

The Zupply framework presented in Chapters~\ref{ch:zupply-design} and~\ref{ch:zupply_implementation}, enabled the maintenance of authenticated \gls{dag}-structured data, with a focus on supply chain management (\glspl{scm}). However, \glspl{dag} are widely used in various domains where representing data flow, dependencies, and sequential relationships is essential. The \gls{dag} structure facilitates mapping the progression between different stages while preserving their ordering and interdependencies. A notable example is version control systems (\glspl{vcs}), such as Git~\cite{gitonline2023} and Mercurial~\cite{mercurial}, which leverage \glspl{dag} to represent the evolution history of a project through its commits. Extending the Zupply framework to support such applications remains a promising direction for future work.
	
Moreover, the arithmetic circuits used in the Zupply framework can be optimized by replacing the standard cryptographic hash function we used (i.e., SHA-256) with algebraic hash functions such as Poseidon \cite{Grassi2021Poseidon}, which require fewer \gls{r1cs} constraints. Moreover, the arithmetic circuits can be instantiated over smaller fields while running the \gls{zksnark} algorithms over larger fields by leveraging the methods proposed in \cite{Diamond2023Towers} and \cite{Gong2024}.


Future work can also explore alternative decentralized cloud storage (\gls{dcs}) platforms, such as PriCloud~\cite{Kopp2021PriCloud}, which provides enhanced user privacy and stronger unlinkability between users and their stored files. However, it is important to consider that different \gls{dcs} platforms may adopt varying standards for content addressing. Additionally, the \gls{ipfs} protocol~\cite{Benet2014} does not inherently guarantee data availability or persistence unless nodes are explicitly pinned and maintained. Therefore, investigating \gls{dcs} alternatives that offer more robust and persistent storage solutions presents another promising direction for future research.
