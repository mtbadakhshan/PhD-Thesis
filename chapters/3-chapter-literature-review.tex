%======================================================================
\chapter{Literature Review}\label{ch:lit_review}
%======================================================================

\section*{Declaration of Contributions}
Section~\ref{sec:lit_additive_fft} of this chapter includes material from an unpublished work that I am a co-author. Additionally, other sections incorporate content from published works ~\cite{Badakhshan2023Outline} (\copyright \hspace{.1em}2023 IEEE), \cite{Badakhshan2025AuroraZupply} (\copyright \hspace{.1em}2025 IEEE), and an unpublished pre-print~\cite{Badakhshan2024Zupply}, all of which I authored under the supervision of Professor Guang Gong.

%\section{Post Quantum Secure zkSNARKs}
%
%Among these \glspl{zksnark}, Fractal features the fastest verifier algorithm, Ligero boasts the fastest prover algorithm, and Aurora achieves the smallest proof size. While STARK employs Algebraic Intermediate Representation (AIR), which is designed for converting the execution trace of a program into algebraic representations (e.g., polynomials), the others rely on Rank-1 Constraint System (R1CS) \cite{Gong2024}, which is suited for arithmetic circuit and is preferred in cryptography related applications, as many privacy-preserving solutions rely on zero-knowledge proofs of knowledge of the preimage of a leaf in a Merkle hash tree constructed by the circuit of a hash algorithm (e.g., SHA-256). Furthermore, zkSNARK-based post-quantum digital signature schemes typically involve proving knowledge of the secret key for a symmetric-key encryption algorithm (e.g., AES) represented as an arithmetic circuit.
%
%\subsubsection{Ligero}
%
%\subsubsection{STARK}
%
%\subsubsection{Aurora}
%
%\subsubsection{Fractal}
%
%\subsubsection{Virgo}
%
%\subsubsection{Spartan}



\section{Additive FFT Algorithms}\label{sec:lit_additive_fft}
The fast Fourier transform (\gls{fft}) over additive groups, known as additive \gls{fft}, over finite fields was developed in the late 1980s. Wang and Zhu~\cite{WangZhu1988} first introduced this concept in 1988, followed independently by Cantor~\cite{Cantor1989FFT} in 1989. These algorithms evaluate polynomials at the roots of the vanishing polynomial corresponding to a subspace or an affine subspace. In~\cite{Cantor1989FFT}, Cantor introduced an \gls{fft} algorithm for the evaluation of a polynomial $f(x)$ of degree less than $n = 2^m$ over an $m$-dimensional affine subspace of the field (or subfield) $\mathbb{F}_{2^k}$, with $k = 2^\ell$ for some $\ell$. This foundational work was later generalized to accommodate any arbitrary values of $k$ by von zur Gathen and Gerhard~\cite{zurGathenFFT}, although their approach incurs a greater operational cost. Subsequently, Gao and Mateer~\cite{Gao2010FFT} proposed two algorithms for computing the additive \gls{fft} utilizing the concept of Taylor expansion. The first algorithm is applicable to any arbitrary $k$ and requires fewer operations compared to the method developed by von zur Gathen and Gerhard. Their second algorithm, designed for the additive \gls{fft} of length $2^m$, where $m$ is a power of 2, matches Cantor’s in the number of multiplications while reducing the number of additions. 


In 2014, Lin, Chung, and Han~\cite{LCH-basis2014} designed a new polynomial basis, referred to as the LCH basis, constructed using the subspace polynomials over $\mathbb{F}_{2^k}$ to implement \gls{fft}. Their evaluation algorithm takes the coefficients of a polynomial in the LCH basis as input and operates with a complexity of $O(n \log n)$ additions and $O(n \log n)$ multiplications. Later, Lin et al.~\cite{LCH-conv2016} addressed the challenge of converting between the LCH basis and the monomial basis. They proposed algorithms that perform this conversion with $O(n \log n \log \log n)$ additions and no multiplications, specifically when the subspace is generated by the Cantor special basis. The LCH additive \gls{fft} for Cantor bases has since been applied to binary polynomial multiplication in several works~\cite{LCH-Fast_Mult2018,LCH-Frobenius2018}. It is important to note that when applying the additive \gls{fft} using these algorithms, for a polynomial with $(t+1)$ coefficients (i.e., of degree $t$), if $t+1$ is less than $2^m$, zero-padding is required to make its size $2^m$. In~\cite{BernsteinChouSchwabe2013}, Bernstein et al. generalized the algorithm of Gao-Mateer a to avoid wasting time on manipulating coefficients that are known to be zero. Later, Bernstein and Chou~\cite{BernsteinChou2014} introduced additional improvements to the algorithm by considering the Cantor basis and the use of tower field construction. These improvements are primarily discussed for \gls{fft} of lengths up to $64$.



%----------------------------------------------------------------------

\section{Blockchain Based Privacy Preserving Protocols} \label{sec:lit_Blockchain_based_preserving_protocols}

Bitcoin \cite{nakamoto2008bitcoin} is the first blockchain platform provides an authenticated tamper-proof record of transactions maintained in a peer-to-peer network. Transactions on Bitcoin are linked together. Many research works such as \cite{Reid2013, Ron2013, Meiklejohn2016} showed that analyzing the transaction graphs, values, and dates of transactions helps to retrieve information that leads to attribute ownership of Bitcoin addresses. Consequently, the identity of the owners could be revealed easier than was expected. Researchers have proposed privacy-focused solutions, such as CoinJoin \cite{coinjoin}, in response to privacy concerns. However, these solutions do not necessarily provide complete privacy guarantees.

Zerocash \cite{zcash-proc} is a cryptocurrency that shares similarities with Bitcoin in terms of its blockchain technology. However, Zerocash uses zero-knowledge proofs (\glspl{zkp}) to break the link between the input of a new transaction and the unspent transaction output (\gls{utxo}) of an existing transaction on the blockchain, thereby providing enhanced privacy. Zerocash can be implemented on top of Bitcoin or other blockchains, and the use of private transactions is optional.
The core of Zerocash's privacy features centers around two primary algorithms: \textsf{Mint} and \textsf{Pour}. The \textsf{Mint} algorithm allows a user to convert any coin value they own into a committed coin, which is then securely stored within a Merkle tree \cite{Merkle1980}. This tree maintains all such commitments. When a user wishes to transfer the committed coins, they utilize the \textsf{Pour} algorithm. This process enables the user to spend up to two of their committed coins, creating up to two new coin commitments or making them directly spendable as public values. The recipient of these coins is specified within the algorithm, ensuring that only they can spend these new commitments. The brilliance of the \textsf{Pour} algorithm lies in its ability to maintain the anonymity of both the sender and receiver. It does so by concealing the commitments being spent and encrypting the details of the new commitments with the receiver's public keys. Additionally, it breaks the traceable link between the spent coins and the newly created commitments, further enhancing privacy.



With the possibility of implementing the idea of smart contracts on Blockchain platforms (such as Ethereum~\cite{ethereum}), we can see an increasing growth in the implementation of various financial decentralized applications (\glspl{dapp}) on Blockchain. Transferring financial activities to the blockchain platforms without paying attention to the privacy of users can have worrying consequences. However, Zerocash lacks support for smart contracts, which limits its adaptability for diverse applications, such as SCM. To redesign Zerocash to accommodate SCM application, one would need to develop a new layer-1 blockchain. Achieving the necessary widespread adoption for this blockchain to maintain its security is a considerable challenge, rendering such a redesign impractical.
Because of this, many efforts have been made to provide methods to protect people's privacy in the blockchain. Such as,  Hawk \cite{Hawk}, Zexe \cite{ZEXE}, ZeeStar \cite{ZeeStar}, etc. These methods are general approaches that aim to preserve the privacy of users.

Hawk \cite{Hawk} is a smart contract compiler that enables users to interact with a smart contract in a secure manner. Hawk developers implemented \glspl{zksnark} protocols on Ethereum smart contracts. Hawk incorporates the privacy-enhancing \textsf{Mint} and \textsf{Pour} algorithms, originally from Zerocash, into smart contracts. Users wishing to execute a computation (e.g., an auction) first commit their native tokens (e.g., ETH) using the \textsf{Mint} algorithm in the Hawk smart contract. They then utilize the \textsf{Pour} algorithm to allocate a portion of these committed tokens for specific computations. Subsequently, users reveal their private inputs to a trusted manager, who executes the computation off-chain and publishes the results on the blockchain. This process ensures the confidentiality of the users' inputs while enabling the  execution of complex computations. 
This allows the use of native cryptocurrency on smart contracts anonymously by hiding money transfers within a smart contract. However, this approach is not suitable for many \glspl{dapp}. For example, where users need to transfer other tokens on Ethereum anonymously, or they want to employ functionalities of smart contracts other than token transformations (\textsf{Pour}). Additionally, Hawk requires a protocol manager.

Bowe et. al presented Zexe \cite{ZEXE} that addresses privacy and scalability in Ethereum blockchain. They propose decentralized private computation (DPC) scheme which extends Zerocash. Zexe leverages DPC to enable offline computation. The offilie computations produces publicly-verifiable transactions that prove correctness of these offline executions. However, it requires cryptographic expertise for implementing new applications~\cite{ZeeStar} and a separate trusted setup for each application~\cite{VERI-ZEXE}. Xiong et al. \cite{VERI-ZEXE} proposed a new DPC scheme called VeriZexe which needs only one single universal setup to be able to support any number of applications.

ZeeStar \cite{ZeeStar} is a compiler that converts a smart contract into an anonymous smart contract where computations are performed off the blockchain on the user side. The encrypted values are then assigned to variables on the smart contract, ensuring that no one knows the actual values of the variables. To verify the accuracy of the computation done by the user, a zero-knowledge proof is uploaded along with the newly encrypted variables. This means that the new state of the smart contract is computed off chain and its correctness is verified by nodes on the Ethereum platform. To achieve this capability, the user must rewrite the smart contract and add privacy annotations to the code. Accordingly, the user specifies which parts of the code should be private and which parts can be public. 
ZeeStar \cite{ZeeStar} provides more applicability than Hawk \cite{Hawk}. However, it still needs to verify the transaction sender according to its address to determine if the sender has permission to execute the rest of the called function. Therefore, using address-based authentication to verify the transaction sender of a smart contract can lead to significant information leakage. 

Narula et. al \cite{zkLedger2018} proposed zkLedger which is a private but also audiable transaction ledger. Their focus is on banks that want their transactions be encrypted; but, let regulators have an insight into bank assets and trades. Auditor can query the bank and have a verifiable proof that the answer actually is true. zkLedger leverages Peterson commitment \cite{petersen1997convert} for committing values. Peterson commitment enables linear functions over transaction values. For example, ratios, percentages, sums, averages, etc. zkLedger leverages zero-knowledge proofs to ensure that banks have valid transactions, i.e., they had consent to transfer the asset, they had enough asset to transfer, and the assets are neither created or destroyed.

\section{Supply Chain Management} \label{sec:lit_SCM}

A supply chain management (\gls{scm}) is a system used by a business and its suppliers to make and deliver certain products to the customers. An \gls{scm} consists of different components such as inventory management, transportation and logistics, and supply chain analytics. In \gls{scm} schemes, it is necessary to monitor the history of each product to gain visibility of the product's flow from producers to consumers. 
At each stage, entities possessing the product must have access to its history to trace its origins. This traceability ensures the product's quality and compliance with ethical standards \cite{SUN2019658, Guo2015, Ackerman2014, Regattieri2007Traceability}. 
Data is either captured by the Internet of things (\gls{iot}) sensors connected to products or entered manually. In the first case, the data is transmitted via different communication protocols, such as  radio-frequency identification (\gls{rfid})  \cite{Tian2016, TAN2022RFID}, ZigBee sensors \cite{Safaric2006}, or Bluetooth low energy (\gls{ble}) \cite{bluetooth2024specification} to receivers. Then, the manually entered or sensor-captured data are aggregated for real-time analysis or future research. The collected data provides valuable information for business owners. Also, it eases the mind of the final customers about their product. For example, whether a wild-caught salmon is really caught from a lake, river, etc. or it is just a farm-raised salmon.  


In supply chains, the ownership of products often changes as the products move through the chain. To accommodate this, ownership models has been developed for \gls{rfid} tags that enables the current owner to authenticate the tag and transfer its ownership. For example, numerous research works, such as \cite{Bi2023, Cherneva2021, yang2017privacy}, focus on transferring the ownership of \gls{rfid} tags while maintaining privacy. However, the process of uploading measured data from the products still can potentially compromise the privacy of the product owners.
It is crucial to protect personal details and trade secrets within the supply chain. This includes maintaining confidentiality about trading partners and involvement in specific supply chains.
Moreover, an \gls{scm} necessitates enabling trusted interoperability among various entities, not only within a predetermined supply chain but also fostering cooperation between different supply chains \cite{Pan2021}. This is particularly crucial in the face of unforeseen situations (e.g., pandemics and natural disasters) where the transiliency of a supply chain becomes paramount \cite{Craighead2020, Sarkis2021}. Transiliency is defined as the ability to both return to the original form and simultaneously undergo transformation (through adaptation and innovation) when facing disruptions, is essential for maintaining continuity and efficiency in supply chains \cite{Craighead2020}. An \gls{scm} requires the establishment of trustworthy collaborations with Micro, Small and Medium Enterprises (\glspl{msme}), which are defined as firms with fewer than 300 employees~\cite{IFCMSME2023}. The involvement of \glspl{msme} in \glspl{scm} offers several benefits, including transparency, traceability,  resilience, and sustainability \cite{Winter2023SMEs, Arend2005SME, UNSME2022}.

Traditional \gls{scm} systems have encountered numerous challenges, particularly in achieving comprehensive transparency and visibility across their operations. The absence of effective traceability measures, lead to significant risks of fraud and product mislabeling. For instance, a study conducted by Shehata et al. (2019) reports a mislabeling rate of 32.3\% among targeted finfish species within the supply chain in southern Ontario, Canada~\cite{Shehata2019mislabelling}. Furthermore, research by Lechmere (2016) shows that up to one in five bottles of fine wine in the market may be counterfeit  \cite{Lechmere2016}. These findings highlight the critical need for enhanced traceability and verification mechanisms within an \gls{scm} to mitigate such risks. The heterogeneity of \gls{scm} platforms, the use of independent centralized databases, and the adoption of different data standards by various entities within a supply chain significantly limit transparency and traceability. Additionally, mistrust among supply chain partners further impedes collaboration \cite{Hellani2021Transparency}. Moreover, payments in current  \gls{scm} systems are centrally managed by the owners according to invoices from their business partners, a process that is both error-prone and difficult, requiring entities to trust the centralized authority~\cite{HyperledgerWalmartCasestudy}. The lack of agility in inter-\gls{scm} collaboration also renders these networks vulnerable during crises. For instance, the COVID-19 pandemic's disruption of manufacturing and logistics activities highlighted the need for supply chains to rapidly employ alternative resources. Examples include converting passenger airplanes to carry freight in their belly cargo, consolidating freight, and implementing on-site storage solutions~\cite{Sudan2021}. Furthermore, \glspl{msme} face greater challenges than large companies in integrating with larger \glspl{scm}. They often depend on their larger partners for relation-based rents, profits generated jointly in an exchange relationship \cite{Dyer1998Rents, Arend2005SME}. Additionally, due to their lesser-known reputations, \glspl{msme} frequently struggle to gain the trust of other companies~\cite{Arend2005SME}.

Supply chains' data records have a sequential format since they represent the conditions of a product over time. Furthermore, two distinct supply chains might merge due to various reasons, such as using the same warehouse, the same transportation, or assembly in the manufacturing stage. In contrast, a single supply chain can be divided into two or more sub supply chains for reasons like the distribution of a product to different locations. Consequently, directed acyclic graphs (\glspl{dag}) are an optimal data structure for storing supply chain data records. For instance, a food supply chain can be seen as a \gls{dag}, where each node in the \gls{dag} signifies a data record captured by an entity responsible for moving, storing, or processing batches of food over a period of time. 

\subsection{Blockchain in Supply Chain Management} \label{sec:lit_bc_in_scm}

Applying blockchain technologies to secure \gls{scm} is a promising approach. Blockchain removes the need for trusting third parties; Moreover, the potential benefits that blockchain and \gls{iot} can give to \gls{scm}, such as traceability, transparency, less paperwork and less code of conduct violation and fraud~\cite{Aich2019}.  Blockchain platforms can be divided into two categories: permissioned and permissionless blockchains. 

In permissioned blockchains, the number of participants is limited, all parties are known, and there is no anonymity. Permissioned blockchains, while providing visibility and automation for supply chain owners, impose certain limitations on their partners, who are required to adhere to the owner's platforms, protocols, and standards. This requirement undermines the agility necessary for maintaining a sustainable \gls{scm} system; notably, integrating a new partner into the system demands a time-consuming process to align with these protocols. Furthermore, the cost of permissioned blockchain solutions can be prohibitively expensive. For example, the IBM Food Trust's~\cite{IBM-SCM}  minimum featured solution, accommodating up to five supply chain partners, costs USD 2,000 per month \cite{IBMFoodTrust}. 
 IBM Food Trust leverages Hyperledger Fabric \cite{Fabric} which is a permissioned blockchain and one of the Hyperledger projects hosted by The Linux Foundation. In Hyperledger Fabric, membership service provider (\gls{msp}) registers member who can publish and share information. Permissioned data access is an essential part of IBM Food Trust. Walmart collaborated with IBM in 2016 to build a permissioned blockchain-based system to trace the origin and transportation of food goods \cite{HELO2019242}. Their blockchain platform is also build on Hyperledger Fabric to trace over 25 products from 5 different suppliers. Blockchain technology has enabled Walmart to track a food item from the store back to its source within seconds. For instance, with the use of blockchain, the time required to trace the origin of mangoes in the U.S. has been reduced from 7 days to just 2.2 seconds \cite{Walmart}.  However, the lack of flexibility, challenge the practicality and scalability of permissioned blockchains in dynamic supply chain environments.


Permissionless (public) blockchains provide a fully decentralized and trustless environment, enabling a wide array of global entities to collaborate, with this collaboration facilitated by smart contracts. Smart contracts offer financial guarantees through their transparent and immutable execution, significantly aiding \glspl{msme} in building trust with their partners. By automating contract enforcement, these digital agreements ensure that all parties fulfill their obligations, such as timely payments or delivery of services or products, thereby mitigating risks associated with traditional contracts~\cite{Agapiou2023SmartContracts}. Furthermore, the integration of entities into the \gls{scm} system is facilitated by the transparent rules established by smart contracts, making the \gls{scm} more agile. Moreover, public blockchains offer enhanced security against 51\% attacks. This accessibility, transparency and security pave the way for developing more sustainable solutions, making public blockchains an attractive option for future \gls{scm} innovations.  This category of blockchains can provide a higher level of transparency and robustness. Also, they do not need a centralized party to grant or revoke permissions. Bitcoin and Ethereum are two well-known examples of permissionless blockchains. Permissionless blockchains can let any user participate as an entity in a supply chain. One problem in permissionless blockchains is that data storage on the blockchain is very expensive. So in some approaches data is stored on decentralized data storages like interplanetary file system (\gls{ipfs}) \cite{Benet2014}.

Tracr\texttrademark~\cite{Tracr} is an example of an \gls{scm} that uses the public blockchain Ethereum. Tracr has been introduced by the De Beers Group to monitor diamonds from the mining stage, through cutting and polishing, and finally to the jewellers. This system provides tamper-proof assurance of the diamond's source~\cite{Kshetri2022}. 
Musamih et al. \cite{Musamih2021} proposed blockchain-based system for pharmaceutical supply chain to address counterfeit drugs issue that is one of the consequences of complex healthcare supply chains structures. They use smart contract over Ethereum platform to define different functions of the different stakeholders (entities) in a supply chian. Such as initializing (manufacturing) a Food and Drug Administration (\gls{fda}) approved drug on blockchain. In their approach the manufacturer and distributors can update data captured by \gls{iot} devices on the blockchain. Their scheme provides the capability of uploading product images to the  \gls{ipfs}, where the hash of the image is published on Ethereum blockchain. The smart contract authenticate stakeholders using their Ethereum address.

Salah et al. \cite{Salah2019} proposed an \gls{scm} solution based on the Ethereum Blockchain and \gls{ipfs} to solve the problem of traceability in the agricultural supply chain where it is difficult to track and trace products in centralized controlled supply chain. Accordingly, in the event of contamination, identifying the source will be easier in blockchain based \gls{scm} solution. They employ Ethereum smart contracts to automate and enforce the rules and regulations of the supply chain and execute specific actions automatically when certain conditions are met. In their approach details of the product is captured and saved on \gls{ipfs} as images. The details can be the time-stamped corp growth images. Hash of the stored file in \gls{ipfs} is stored in the smart contract. The authentication of entities in their proposed approach is based on Ethereum address.

Toyoda et al. \cite{Toyoda2017} proposed a novel product ownership management system (POMS). They have implemented POMS on a Ethereum smart contract. In their proposed approach, the manufacturer assign Electronic Product Code (EPC) to each product and write that into the \gls{rfid} tag attached to the product. The product's EPC is registered on the smart contract. Furthermore, functions such as product owner transformation, incentivising entities to follow POMS protocol, and unauthorized party prevention are enabled by their smart contract. Toyoda et al. claimed PMOS system makes counterfeiters' efforts to clone real tags ineffective. 


\subsection{Blockchain Based Privacy Preserving Solutions for Supply Chain Management}\label{sec:lit_pp_bc_scm}

Privacy, especially against public exposure, is crucial in \gls{scm} applications. It encompasses entities' anonymity—ensuring that auditors reviewing the history of a product cannot identify the creator of a specific data record unless the entity chooses to reveal its identity. Additionally, privacy involves unlinkability among entities, meaning that businesses desire to keep their partners and the supply chains they are involved in confidential. Our analysis reveals that all of the privacy preserving SCM solutions reviewed are not practical to be employed as SCM systems on public blockchains. 

\sloppy The importance of privacy preservation in \glspl{scm}  fuels research into permissioned blockchains such as~\cite{Basim2023Privacy, Li2024ProChain} that offer restricted access and rely on a centralized membership service provider~\cite{Fabric2018}. However, preserving the entities' anonymity while verifying the authenticity of the data they have uploaded is an important issue that needs to be resolved in permissionless blockchains. Many \gls{scm} schemes developed over permissionless blockchains, such as those mentioned above, do not preserve the privacy of their participants. Each entity has to interact with the blockchain ledger through a generated address for invoking smart contract functions and covering gas fees with the blockchain's native token (e.g., ETH), which compensate for the processing and validation of transactions. Entities typically obtain the required tokens through exchanges enforcing know your customer (KYC) protocols, or they may acquire them from other recognized addresses, with or without the involvement of an intermediary. This requirement introduces a risk: through detailed analysis of blockchain transactions, it is possible to infer the identity of the entities involved \cite{Victor2020Address, Zhou2022EthereumGraph}. Given the transparency of transactions on permissionless blockchains, these schemes could potentially compromise participants privacy.  Following reviews some privacy preservation methods over public blockchains.


AlTway et al. presented Mesh \cite{altawy2019mesh}, a supply chain solution over Ethereum smart contracts. Mesh uses group signatures to preserve the privacy of participant and employs a forward secrecy approach to preserve the confidentiality of data. Both are  required to protect business secrets. However, Mesh requires a centralized  server to keep the protocol going.  Mesh uses Petersen's group signature \cite{petersen1997convert} to authorize members of a supply chain. For each supply chain, all entities that are involved in that supply chain must build a group. The group must have a group manager which sends a list of identities of all group members to the Mesh Server and the Mesh's SupplyChain smart contract ($C_{SC}$) instance. For each smart contract, a separate instance of $C_{SC}$ should be created. Only the group members of a supply chain are able to upload data to the related $C_{SC}$. The data is stored on Ethereum's Blockchain. To limit the upload access to the group members, Altawy et al. \cite{altawy2019mesh} use Petersen's group signature \cite{petersen1997convert} for members' authorization. Consequently, $C_{SC}$ will be able to verify the membership of an entity without learning the identity of that user. However, it is clear that the transaction is from a known group without knowing which group member is sending the data. Also, Petersen's scheme provides plausible anonymity which means that although group members are locally anonymous, their identity can be revoked by the manager, Mesh Server or coalition of members. 
Although transactions are anonymous, they are still recognizable as originating from a known group, preserving local anonymity of group members.  Mesh utilizes forward secrecy techniques to protect data confidentiality. Forward secrecy, as described in \cite{chen2012communication}, allows any entity to decrypt and view the product's history up to the point of its ownership transfer. However, access to subsequent records is restricted, preventing entities from viewing future transactions or exchanges after the point of transfer. 






DECOUPLES \cite{Maouchi2019DECOUPLES} is an SCM solution utilizing the blockchain presented in \cite{Flymen2020Learn}. This blockchain features a customized block validation mechanism specifically designed for DECOUPLES, rendering it incompatible with established blockchain like Bitcoin \cite{nakamoto2008bitcoin} or Ethereum \cite{ethereum}. In their approach, certificates are issued and signed by a certificate organization (CO) for each participant in the supply chain. These certificates are then sent to the participant through certificate transactions on the blockchain. Certificate holders can subsequently use Schnorr signatures \cite{Schnorr1991Signature} to authenticate certificate ownership in their transactions, which include product information.
Within the framework, Maouchi et al. proposed the product-specific stealth addresses (PASTA) protocol. This protocol employs Stealth addresses \cite{Nicolas2017Stealth}, to ensure that multiple payments made to the same payee are unlinkable, thereby preserving the transaction receiver's anonymity. Furthermore, to anonymize the sender, they utilized multi-layered linkable spontaneous anonymous group (MLSAG) ring signatures \cite{noether2016ring}. Additionally, confidential transaction information is encrypted using the elliptic curve integrated encryption scheme (ECIES) \cite{brown2009standards}, with the public key of the receiving party. 
The DECOUPLE framework does not account for complex supply chain operations, such as the dividing or merging of products. Additionally, the security of a blockchain  is largely contingent on its decentralization, typically achieved through a diverse and distributed network of validators. However, the framework relies on their customized blockchain might impact its decentralization.




zkLedger \cite{zkLedger2018} is a transaction ledger designed to offer both privacy and auditability, making it ideal for banks seeking to encrypt their transactions while granting regulators access to their assets and deals. It allows auditors to request and receive verifiable confirmation from the banks, ensuring the accuracy of their responses. zkLedger utilizes a columnar ledger structure for secure and private transaction management. In this design, each column represents a different bank, while each row details a transaction where assets are transferred between banks. To execute a transaction, the sending bank makes two commitments using Pedersen commitments \cite{petersen1997convert}: one for the assets deducted from its account and another for the assets credited to the recipient's account. These commitments are then placed in their respective columns. To enhance unlinkability, the sending bank also commits to zero for all other banks, effectively anonymizing the transaction by obscuring the involvement of banks not participating in the transaction. This approach ensures that the details of the transaction remain private, except to the parties involved and authorized auditors. zkLedger incorporates three crucial types of cryptographic proofs to maintain integrity and privacy: (1) \textit{Proof of Balance} ($\pi^{B}$) ensures that the total assets transferred in a transaction equate to zero, indicating no creation or destruction of assets, without disclosing the sender. (2) \textit{Proof of Assets} ($\pi^{A}$) verifies that the spending bank has enough assets for the transfer. (3) \textit{Proof of Consistency} ($\pi^{C}$) guarantees the integrity of the ledger by preventing the inclusion of invalid data that could compromise the ledger's verifiability or an auditor's ability to validate transactions. Pedersen commitments enable banks to perform secure statistical analyses—like calculating sums, averages, and variances—on their assets for auditors without revealing specific transaction details. This ensures both transparency and privacy, allowing banks to verify their financial health while maintaining the confidentiality of transactions.
zkLedger's design mandates that all transactions be recorded directly on the ledger. This requirement poses challenges for SCM applications that necessitate detailed documentation of product histories. Deploying zkLedger on public blockchains, such as Ethereum, could incur prohibitive costs due to the extensive information storage required. Additionally, utilizing zkLedger as an Ethereum smart contract would inadvertently reveal the addresses of entities uploading data records, potentially compromising their privacy. Furthermore, zkLedger requires unanimous consent among all participating banks for adding or removing an entity (column) from the ledger. This consensus mechanism, combined with the increased complexity and time demands for transaction processing, broadcasting, and verification with more participants, significantly decelerates the process. These constraints impede zkLedger's scalability and adaptability in SCM scenarios, essentially preventing smaller entities from easily integrating into the network. Lastly, zkLedger recommends that auditors maintain commitment caches to expedite the verification of column value sums. Without these caches, auditors are compelled to process the entire ledger for transaction confirmation, regardless of their relevance to the audited entity or asset. This necessity places a considerable burden on end customers, particularly those who do not consistently monitor the ledger







